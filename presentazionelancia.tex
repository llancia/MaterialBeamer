\documentclass{beamer}

\usepackage[english]{babel}
\usepackage{graphicx,hyperref,url, sapbeamer}
\usepackage{braket}
\usepackage{euler}
\usepackage{listings}

\lstdefinestyle{customsql}{
  belowcaptionskip=1\baselineskip,
  breaklines=true,
  xleftmargin=\parindent,
  language=SQL,
  showstringspaces=false,
  basicstyle=\footnotesize\ttfamily,
  keywordstyle=\bfseries\color{green!40!black},
  commentstyle=\itshape\color{purple!40!black},
  identifierstyle=\color{blue},
  stringstyle=\color{orange},
}
\lstset{escapechar=@,style=customsql}

\usefonttheme{professionalfonts} % using non standard fonts for beamer
\usefonttheme{serif}
\usepackage[no-math]{fontspec}
\usepackage{xcolor}
\setmainfont[Mapping=tex-text,  
    BoldFont={Frutiger65-Bold.ttf}, 
    ItalicFont={frutiger56-i.ttf},
    BoldItalicFont={frutiger66-bi.ttf}]{frutiger55.ttf}

% The title of the presentation:
%  - first a short version which is visible at the bottom of each slide;
%  - second the full title shown on the title slide;
\title[Fb Tao]{
   Tao }

% Optional: a subtitle to be dispalyed on the title slide
\subtitle{Distributed Data Store for the Social Graph}

% The author(s) of the presentation:
%  - again first a short version to be displayed at the bottom;
%  - next the full list of authors, which may include contact information;
\author[L. Lancia \& G]{
  Lorenzo Lancia \\ G
  } 
  
%\titlegraphic{\includegraphics[width=\textwidth]{atac-logo}}

% The institute:
%  - to start the name of the university as displayed on the top of each slide
%    this can be adjusted such that you can also create a Dutch version
%  - next the institute information as displayed on the title slide
\institute[Sapienza Università di Roma]{
  Master Degree in Data Science \\
  Sapienza Università di Roma}

% Add a date and possibly the name of the event to the slides
%  - again first a short version to be shown at the bottom of each slide
%  - second the full date and event name for the title slide
\date[\today]{
 \today}




\providecommand{\di}{\mathop{}\!\mathrm{d}}
\providecommand*{\der}[3][]{\frac{d\if?#1?\else^{#1}\fi#2}{d #3\if?#1?\else^{#1}\fi}} 
 \providecommand*{\pder}[3][]{% 
    \frac{\partial\if?#1?\else^{#1}\fi#2}{\partial #3\if?#1?\else^{#1}\fi}% 
  }
\begin{document}

\begin{frame}
  \titlepage
\end{frame}

\begin{frame}
  \frametitle{Indice}

  \tableofcontents
\end{frame}

% Section titles are shown in at the top of the slides with the current section 
% highlighted. Note that the number of sections determines the size of the top 
% bar, and hence the university name and logo. If you do not add any sections 
% they will not be visible.
\section{Introduction}
\begin{frame}
  \frametitle{Introduction}
  \begin{itemize}
  \item Aim of this project was to build a simple relational database
    and executing some queries.

    \item We used open data about public transport network of city of Rome
    provided by ATAC. \url{http://www.agenziamobilita.roma.it/it/progetti/open-data/dataset.html}
  
   \item The DBMS used is MySql\footnote{Ver 14.14 Distrib 5.7.11}
  \end{itemize}

 \end{frame}
\section{Open Data}
\begin{frame}
  \frametitle{Open Data}
  ATAC provides the data in the GTFS format. 
  
 \begin{block}{What is GTFS?}
The General Transit Feed Specification (GTFS) defines a common format
for public transportation schedules and associated geographic
information. \footnote[frame]{\url{https://developers.google.com/transit/gtfs/\#how-do-i-start
  }}
\end{block}

Unfourtunatly not all tables required in GTFS standard are available in the
ATAC dataset.
 \end{frame}
\begin{frame}[fragile]
\frametitle{File Used}
\begin{table}
  \centering
\small
  \begin{tabular}{p{3cm} p{8cm} }
 \verb|stops.txt| & Individual locations where vehicles pick up or drop off passengers.\\
\verb|routes.txt| & Transit routes. A route is a group of trips that are displayed to riders as a single service.\\
\verb|trips.txt|	& Trips for each route. A trip is a sequence of two or more stops that occurs at specific time.\\
\verb|stop_times.txt| & Times that a vehicle arrives at and departs
                        from individual stops for each trip.\\
\verb|calendar_dates.txt| & Exceptions for the service IDs defined in
                            the \verb!calendar.txt! file. If  \verb!calendar_dates.txt! includes ALL dates of service, this file may be specified instead of \verb!calendar.txt!.\\
  \end{tabular}
  \caption{Data Set files}
  \label{tab:tab1}
\end{table}
\end{frame}
\section{Schema}
\begin{frame}[fragile]
  \frametitle{Table routes}
  \begin{lstlisting}
CREATE TABLE `routes` (
  `route_id` varchar(5) DEFAULT NULL,
  `agency_id` varchar(8) DEFAULT NULL,
  `route_short_name` varchar(5) DEFAULT NULL,
  `route_long_name` varchar(10) DEFAULT NULL,
  `route_type` int(1) DEFAULT NULL,
  `route_color` varchar(6) DEFAULT NULL,
  `route_text_color` int(6) DEFAULT NULL,
  KEY `route_id` (`route_id`)
) ENGINE=InnoDB DEFAULT CHARSET=utf8;
  \end{lstlisting}  
\end{frame}

\begin{frame}[fragile]
  \frametitle{Table stops}
  \begin{lstlisting}
CREATE TABLE `stops` (
  `stop_id` varchar(5) NOT NULL DEFAULT '',
  `stop_name` varchar(42) NOT NULL DEFAULT '',
  `stop_lat` decimal(26,10) DEFAULT NULL,
  `stop_lon` decimal(26,10) DEFAULT NULL,
  `location_type` int(1) DEFAULT NULL,
  `parent_station` varchar(5) DEFAULT NULL,
  PRIMARY KEY (`stop_id`,`stop_name`),
  KEY `stop_id` (`stop_id`)
) ENGINE=InnoDB DEFAULT CHARSET=utf8;
  \end{lstlisting}  
\end{frame}


\begin{frame}[fragile]
  \frametitle{Table trips}
  \begin{lstlisting}
CREATE TABLE `trips` (
  `route_id` varchar(5) DEFAULT NULL,
  `service_id` varchar(9) DEFAULT NULL,
  `trip_id` varchar(11) DEFAULT NULL,
  `direction_id` int(1) DEFAULT NULL,
  `shape_id` bigint(10) DEFAULT NULL,
  KEY `trip_id` (`trip_id`),
  KEY `route_id` (`route_id`),
  CONSTRAINT `trips_ibfk_1` FOREIGN KEY (`route_id`) REFERENCES `routes` (`route_id`)
) ENGINE=InnoDB DEFAULT CHARSET=utf8;
  \end{lstlisting}  
\end{frame}

\begin{frame}[fragile]
  \frametitle{Table times}
  \begin{lstlisting}
CREATE TABLE `times` (
  `trip_id` varchar(11) DEFAULT NULL,
  `arrival_time` time DEFAULT NULL,
  `departure_time` time DEFAULT NULL,
  `stop_id` varchar(5) DEFAULT NULL,
  `stop_sequence` int(11) DEFAULT NULL,
  KEY `trip_id` (`trip_id`),
  KEY `stop_id` (`stop_id`),
  CONSTRAINT `times_ibfk_1` FOREIGN KEY (`trip_id`) REFERENCES `trips` (`trip_id`),
  CONSTRAINT `times_ibfk_2` FOREIGN KEY (`stop_id`) REFERENCES `stops` (`stop_id`)
) ENGINE=InnoDB DEFAULT CHARSET=utf8;
  \end{lstlisting}  
\end{frame}

\begin{frame}[fragile]
  \frametitle{Table calendar}
  \begin{lstlisting}
CREATE TABLE `calendar` (
  `service_id` varchar(9) DEFAULT NULL,
  `date` date DEFAULT NULL,
  `exception_type` int(1) DEFAULT NULL
) ENGINE=InnoDB DEFAULT CHARSET=utf8;
\end{lstlisting}
\end{frame}
\section{Query}
\begin{frame}[fragile]
\frametitle{Query}
1) List all routes that stops at "Policlinico"
\begin{columns}
\column{0.5\textwidth}
\begin{lstlisting}
SELECT DISTINCT route_id
FROM times join stops join trips 
ON times.`stop_id`=stops.`stop_id` 
AND trips.`trip_id`=times.`trip_id`
WHERE stops.`stop_name` = "POLICLINICO"
\end{lstlisting}
\column{0.5\textwidth}
results:
N2L
N13
490
495
61
N2
649
2
19
3
N10
88
N11
MEB
MEB1
\end{columns}
\end{frame}
\begin{frame}[fragile]
2) List all routes that a stop at a stop containing the word "DE
LOLLIS" 
\begin{columns}
\column{0.5\textwidth}
\begin{lstlisting}
SELECT DISTINCT route_id, stop_name 
FROM times JOIN stops JOIN trips 
ON times.`stop_id`=stops.`stop_id` 
AND trips.`trip_id`=times.`trip_id`
WHERE stops.`stop_name` LIKE "%DE LOLLIS%"
\end{lstlisting}
\column{0.5\textwidth}
results: 
\begin{verbatim}
C3 DE LOLLIS- IRPINI
N10 DE LOLLIS- IRPINI
492 DE LOLLIS- IRPINI
C3 DE LOLLIS- VERANO
N10 DE LOLLIS- VERANO
C2 DE LOLLIS- VERANO
492 DE LOLLIS- VERANO
2 VERANO- DE LOLLIS
19 VERANO- DE LOLLIS
3 VERANO- DE LOLLIS
71 VERANO- DE LOLLIS
\end{verbatim}
\end{columns}
\end{frame}
\begin{frame}[fragile]
 3) List the stops of the autobus 445
 \begin{columns}
\column{0.5\textwidth}
   \begin{lstlisting}
SELECT DISTINCT stop_name
FROM stops, times, trips
WHERE stops.stop_id = times.stop_id
AND trips.trip_id = times.trip_id
AND `route_id` = "445"
\end{lstlisting}
\column{0.5\textwidth}
\small
\begin{verbatim}
BOLOGNA
VENTUNO APRILE- VILLA RICOTTI
VENTUNO APRILE- NARDINI
LANCIANI- BOLDETTI
LANCIANI- DE PETRA
MONTI TIBURTINI- NOMENTANA
MONTI DI PIETRALATA
CURIONI- DE LORENZO
CURIONI- COLLINA LANCIANI
CURIONI
CURIONI- PENTA
CURIONI- REPOSSI
LARGO LANCIANI
LANCIANI- WINCKELMANN
VENTUNO APRILE- RICOTTI
CARACI- MINISTERO INFRASTRUTTURE TRASPORTI
\end{verbatim}
 \end{columns}
\end{frame}

\begin{frame}[fragile]
 4) How many trains of MEB runs in a day?
 \begin{columns}
\column{0.5\textwidth}
   \begin{lstlisting}
SELECT count(trip_id), calendar.date
FROM trips join calendar
ON trips.service_id = calendar.service_id
WHERE route_id = "MEB"
GROUP BY calendar.date 
LIMIT 7
\end{lstlisting}
\column{0.5\textwidth}
\small
\begin{verbatim}
count(trip_id) date
375 2016-03-14
375 2016-03-15
375 2016-03-16
375 2016-03-17
400 2016-03-18
267 2016-03-19
241 2016-03-20
\end{verbatim}
 \end{columns}
\end{frame}

\begin{frame}[fragile]
 5) Which line departs from ``VERANO'' after 5pm
 \begin{columns}
\column{0.5\textwidth}
   \begin{lstlisting}
SELECT route_id, departure_time
FROM trips JOIN times JOIN stops 
ON times.stop_id = stops.stop_id 
AND times.trip_id = trips.trip_id
WHERE stop_name = "VERANO"
AND departure_time = (
SELECT MIN(`departure_time`) 
FROM times JOIN stops 
ON times.stop_id = stops.stop_id 
WHERE stop_name = "VERANO" 
AND `departure_time`>"17:00:00")
\end{lstlisting}
\column{0.5\textwidth}
\small
\begin{verbatim}
route_id departure_time
3 17:01:00
71 17:01:00
163 17:01:00
545 17:01:00
542 17:01:00
\end{verbatim}
 \end{columns}
\end{frame}

\begin{frame}[fragile]
6) Which are the three most frequented stops on sunday?
 \begin{columns}
\column{0.5\textwidth}
   \begin{lstlisting}
SELECT  stop_name,  count(times.trip_id) AS cnt
FROM times JOIN stops JOIN calendar JOIN trips
ON times.stop_id = stops.stop_id
AND calendar.service_id = trips.service_id
AND trips.trip_id = times.trip_id
WHERE date = "2016-03-20"
GROUP BY stop_name ORDER BY cnt DESC LIMIT 3
\end{lstlisting}
\column{0.5\textwidth}
\small
\begin{verbatim}
TERMINI 4158
PIAZZA VENEZIA 2770
CONCA D'ORO 2493
\end{verbatim}
 \end{columns}
\end{frame}


\begin{frame}[fragile]
 7) List all of stops served by nocturne bus
 \begin{columns}
\column{0.5\textwidth}
   \begin{lstlisting}
SELECT distinct stop_name, route_id 
FROM trips, times, stops 
WHERE trips.trip_id = times.trip_id 
AND stops.stop_id = times.stop_id 
AND route_id LIKE "N%"
\end{lstlisting}
\column{0.5\textwidth}
\small
\begin{verbatim}
BATTISTINI- SORIA N1
BOCCEA- BATTISTINI N1
BOCCEA- BRA N1
BOCCEA- VAL CANNUTA N1
BOCCEA- GREGORIO TREDICESIMO N1
BOCCEA/URBANO SECONDO N1
BOCCEA- GALEOTTI N1
CIRCONVALLAZIONE CORNELIA- BOLOGNINI N1
CIRCONVALLAZIONE CORNELIA- AURELIA N1
...
CAVE ARDEATINE 	N9
MARMORATA- VANVITELLI	N9
ARA COELI- PIAZZA VENEZIA	N9
TERME DIOCLEZIANO	N9
\end{verbatim}
 \end{columns}
\end{frame}

\begin{frame}[fragile]
 8) List all routes that connects Rebibbia to Tiburtina Station
 \begin{columns}
\column{0.5\textwidth}
   \begin{lstlisting}
SELECT DISTINCT route_id
FROM times t JOIN stops s JOIN trips tr 
ON t.`trip_id` = tr.trip_id 
AND s.stop_id = t.stop_id
WHERE s.stop_name LIKE "%REBIBBIA%"
AND tr.route_id = ANY (
SELECT distinct tr.route_id 
FROM times t JOIN stops s JOIN trips tr 
ON t.trip_id = tr.trip_id AND s.stop_id = t.stop_id
WHERE s.stop_name = "TIBURTINA" 
OR s.stop_name = "STAZIONE TIBURTINA"
)
\end{lstlisting}
\column{0.5\textwidth}
\small
\begin{verbatim}
120F
163
N2
N23
MEB
\end{verbatim}
 \end{columns}
\end{frame}

\end{document}
